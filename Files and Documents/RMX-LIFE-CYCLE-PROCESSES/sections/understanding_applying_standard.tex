\section{UNDERSTANDING AND APPLYING THIS STANDARD}

	\subsection{RELATIONSHIP BETWEEN SYSTEMS AND SOFTWARE}
	\begin{adjustwidth}{0.5em}{0pt}

		This standard establishes a strong link between a system and its software. It is based upon the general principles of systems engineering. Software is treated as an integral part of the total system and performs certain functions in that system. This is implemented by extracting the software requirements from the system requirements and design, producing the software, and integrating it into the system. 

		It is a fundamental premise that software always exists in the context of a system, even if the system consists of only the processor upon which the software is executed. Therefore, a software product or service is always treated as one item in a system. 

		For example, this standard makes a distinction between system requirements analysis and software requirements analysis, because, in the general case, system architectural design will allocate the system requirements to various items of the system and software requirements analysis will derive software requirements from the system requirements allocated to each software item. 

		Note that in some cases, the non-software items of a system may be so minimal that it may not be necessary to perform distinct system and software analyses.

	\end{adjustwidth}

	\subsection{ORGANIZATION-LEVEL AND PROJECT-LEVEL ADOPTION}
	\begin{adjustwidth}{0.5em}{0pt}

		Modern software businesses strive to develop an exhaustive set of software life cycle processes that are applied repeatedly to the software projects of the business. Therefore, this standard is intended to be useful for adoption at either the organization level or at the project level. 

		An organization would adopt this standard and supplement it with appropriate procedures, practices, tools and policies. 

		A software project of the organization would conform to the organization's processes rather than conform directly to this standard. 

		In some cases, projects may be executed by an organization that does not have an appropriate set of processes adopted at the organizational level. Such a project may apply the provisions of this standard directly to the project itself.

	\end{adjustwidth}

	\subsection{TEMPORAL RELATIONSHIPS AMONG PROCESSES}
	\begin{adjustwidth}{0.5em}{0pt}

		Within this standard, the processes and their activities and tasks are arranged in a sequence suitable for exposition. This positional sequence does not prescribe or dictate any time-dependent sequence. 

		For lack of consensus on or use of a universal time-dependent sequence, the reader may select and order the processes, activities, and tasks as appropriate and effective. 

		This standard encourages iteration among the activities and recursion within an activity to offset the effects of any implied sequence of activities and tasks. 

		The organization is ultimately responsible for selecting a life cycle model for the project and mapping the processes, activities, and tasks onto that model, as well as for selecting life cycle models and processes for its own internal management and governance belonging to, or wholly separate from, those used for software projects. 

	\end{adjustwidth}


	\subsection{EVALUATION VS VERIFICATION AND VALIDATION}
	\begin{adjustwidth}{0.5em}{0pt}

		Organizations that are involved in any process of a life cycle conduct evaluations of the products of that task. The Software Verification and Software Validation processes provide the opportunity for additional evaluations. These processes are conducted by the acquirer, the supplier, or an independent party to verify and validate the products in varying depth depending on the project. 

		These evaluations do not duplicate or replace other evaluations, but supplement them. Additional opportunities for evaluation are provided by the Software Review, Software Audit, Software Quality Assurance and the Life Cycle Model Management Processes.

	\end{adjustwidth}


	\subsection{CRITERIA FOR PROCESSES}
	\begin{adjustwidth}{0.5em}{0pt}

		This standard establishes a framework for the life cycle of software. The life cycle begins with an idea or a need that can be satisfied wholly or partly by software and ends with the retirement of the software. The architecture is built with a set of processes and interrelationships among these processes. The determination of the life cycle processes is based upon two basic principles: cohesion and responsibility.\\

		\begin{compactitem}

			\item {\bf Cohesion}: The life cycle processes are cohesive and coupled to the optimum extent deemed practical and feasible \\

			\item {\bf Responsibility}: A process is placed under the responsibility of an organization or role in the software life cycle\\

		\end{compactitem}

	\end{adjustwidth}


	\subsection{DESCRIPTION AND GENERAL CHARACTERISTICS OF PROCESSES}
	\begin{adjustwidth}{0.5em}{0pt}

		The processes within this standard are described in a manner that is meant to facilitate the use of this standard for either, or both, organization-level and project-level adoption. Each process is described in terms of the following attributes:\\

		\begin{compactitem}

			\item {\bf Title}: The title conveys the scope of the process as a whole\\

			\item {\bf Purpose}: The purpose describes the goals of performing the process\\

			\item {\bf Outcomes}: The outcomes express the observable results expected from the successful performance of the process\\

			\item {\bf Activities}: The activities are a list of actions that are used to achieve the outcomes\\

			\item {\bf Tasks}: The tasks are requirements, recommendations, or permissible actions intended to support the achievement of the outcomes\\

		\end{compactitem}

	\end{adjustwidth}


	\subsection{DECOMPOSITION OF PROCESSES}
	\begin{adjustwidth}{0.5em}{0pt}

		Each process within this standard satisfies the criteria described above. For the purpose of clear description,processes are sometimes decomposed into smaller pieces. Some processes are decomposed into activities and/or lower-level processes. A lower-level process is described when the decomposed portion of the process itself satisfies the criteria to be a process. An activity is used when the decomposed unit does not qualify as a process. An activity can be considered as simply a collection of tasks.

		A task is expressed in the form of a requirement, recommendation, or permissible action, intended to support the achievement of the outcomes of a process. For this purpose, this standard carefully employs certain auxiliary verbs (shall, should, and may) to differentiate between the distinct forms of a task. ``Shall'' is used to express a provision required for conformance, ``should'' to express a recommendation among other possibilities, and ``may'' to indicate a course of action permissible within the limits of this standard and common framework.

	\end{adjustwidth}


	\subsection{LIFE CYCLE MODELS AND STAGES}
	\begin{adjustwidth}{0.5em}{0pt}

		The life of a system or a software product can be modeled by a life cycle model consisting of stages. Models may be used to represent the entire life from concept to disposal or to represent the portion of the life corresponding to the current project. The life cycle model is comprised of a sequence of stages that may overlap and/or iterate, as per the project's scope, magnitude, complexity, changing needs and opportunities. 

		Each stage is described with a statement of purpose and outcomes. The life cycle processes and activities are selected and employed in a stage to fulfill the purpose and outcomes of that stage. 

		Different organizations may undertake different stages in the life cycle. However, each stage is conducted by the organization responsible for that stage with due consideration of the available information on life cycle plans and decisions made in preceding stages. Similarly, the organization responsible for that stage records the decisions made and records the assumptions regarding subsequent stages in the life cycle.

		Finally, this standard does not dictate the use of any specific combination or permutation of stages to explicitly and always define any given life cycle. It is left to the organization to determine, by using this standard and its examples and exhaustive breakdown and definition of life cycle processes, those life cycles beneath their own business operations and processes, their own tactical initiatives, activities, and tasks. 

	\end{adjustwidth}