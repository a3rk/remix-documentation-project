\section{RELATIONSHIPS BETWEEN STANDARDS}
\begin{adjustwidth}{0.5em}{0pt}

	The purpose of this informative appendix is to describe relationships to other IEEE standards. The itemized breakdown below lists the processes of this standard, this document specifically. For many of those processes, the list suggests IEEE standards that may be helpful in implementing or executing the process. In each case, a note describes the nature of the relationship. 

	\begin{compactitem}
		
		\item \nameref{subsec:agreement_processes}
		\begin{compactitem}
			
			\item \nameref{proc:acquisition_process}: IEEE Std. 1062, recommendation of useful practices that can be selected and applied during software acquisition.

		\end{compactitem}


		\item \nameref{subsec:organizational_project_enabling_processes}
		\begin{compactitem}
			
			\item \nameref{proc:life_cycle_model_management_process}: IEEE Std. 1074, standard describes an approach for the definition of software life cycle processes.

			\item \nameref{proc:infrastructure_management_process}: IEEE Std. 1175 and 1462, describe the integration of CASE tools into productive software engineering environments, and guidelines for the evaluation and selection of CASE tools, respectively. 

			\item \nameref{proc:quality_management_process}: ISO 90003, standard provides guidance for organizations in the application of ISO 9001:2000 to software. 

		\end{compactitem}


		\item \nameref{subsec:project_processes}
		\begin{compactitem}
			
			\item \nameref{proc:project_planning_process}: IEEE Std. 1058 and 1228, describes the format and content of a software project management plan, and the content of a plan for the software aspects of development, procurement, maintenance, and retirement of a safety-critical system, respectively. 

			\item \nameref{proc:risk_management_process}: IEEE Std. 1540, provides a process for the management of software risk.  

			\item \nameref{proc:measurement_process}: IEEE Std. 982.1, 1045, 1061, and 14143.1, provides a set of measures for forecasting and evaluating the reliability of a software product, a consistent terminology for software productivity measures, a methodology for establishing quality requirements, and the fundamental concepts of a class of measures collectively known as functional size, respectively. 

		\end{compactitem}


		\item \nameref{subsec:technical_processes}
		\begin{compactitem}
			
			\item \nameref{proc:stakeholder_requirements_definition_process}: IEEE Std. 1362, provides guidance on the format and content of a Concept of Operations document, describing characteristics of a proposed system from the user's viewpoint.

			\item \nameref{proc:system_requirements_analysis_process}: IEEE Std. 1233, provides guidance on the development of a system requirements specification and the characteristics and qualities of requirements.

			\item \nameref{proc:system_architectural_design_process}, IEEE Std. 1471, recommends a conceptual framework and content for the architectural description of software-intensive systems.

			\item \nameref{proc:software_maintenance_process}: ISO/IEC 14764, provides guidance on implementing the software maintenance process. 

		\end{compactitem}


		\item \nameref{subsec:software_implementation_processes}
		\begin{compactitem}

			\item \nameref{proc:software_requirements_analysis_process}: IEEE Std. 830, recommends the content and characteristic of a software requirements specification.

			\item \nameref{proc:software_architectural_design_process}: IEEE Std. 1471, recommends a conceptual framework and content for the architectural description of software-intensive systems. 

			\item \nameref{proc:software_detailed_design_process}: IEEE Std. 1016, recommends content and organization of a software design description.

			\item \nameref{proc:software_construction_process}: IEEE Std. 1008, describes an approach to software unit testing.

			\item \nameref{proc:software_integration_process}: IEEE 829, describes the form and content of a basic set of documentation for planning, executing, and reporting software testing. 

			\item \nameref{proc:software_qualification_testing_process}: IEEE 829, describes the form and content of a basic set of documentation for planning, executing, and reporting software testing. 

		\end{compactitem}


		\item \nameref{subsec:software_support_processes}
		\begin{compactitem}

			\item \nameref{proc:software_documentation_management_process}: IEEE Std. 1063, provides requirements for the structure, content, and format of user documentation.

			\item \nameref{proc:software_configuration_management_process}: IEEE Std. 828, specifies the content of a software configuration management plan along with requirements for specific planning activities. 

			\item \nameref{proc:software_quality_assurance_process}: IEEE Std. 730, 1061, and 1465, specifies the format and content of a software quality assurance plan, a methodology for establishing quality requirements and for identifying, implementing, and validating the corresponding measures, and describes quality requirements specifically suitable for software packages, respectively.

			\item \nameref{proc:software_verification_process}: IEEE Std. 1012, describes software verification and validation activities. 

			\item \nameref{proc:software_validation_process}: IEEE Std. 1012, describes software verification and validation activities. 

			\item \nameref{proc:software_review_process}: IEEE Std. 1028, describes five types of software reviews, and procedures for their execution. 

			\item \nameref{proc:software_audit_process}: IEEE Std. 1028, describes five types of software reviews, and procedures for their execution. 

			\item \nameref{proc:software_problem_resolution_process}: IEEE Std. 1044, provides a uniform approach to the classification of anomalies found in software and its documentation. 
			
		\end{compactitem}

		\item \nameref{subsec:software_reuse_processes}
		\begin{compactitem}

			\item For all processes within this section, IEEE Standard 1420.1 and 1517
			
		\end{compactitem}

	\end{compactitem}

\end{adjustwidth}