\section{INTRODUCTION}
\begin{adjustwidth}{0em}{0pt}

	This standard establishes a common framework for life cycle processes, with well-defined terminology, that can be referenced by businesses, organizations, and entities that create, manage, and provide software directly or indirectly, as a part of, or wholly comprising, their purchasable products and services, regardless of the industry in which the business itself is classified or categorized. 

	This standard applies to the acquisition of systems and software products and services, to the supply, development, operation, maintenance, and disposal of software products and the software portion of a system, whether performed internally or externally to an organization. Those aspects of system definition needed to provide the context for software products and services, and a process that can be employed for defining, controlling, and improving software life cycle processes, are provided herein. The processes may also be applied during the acquisition of a system that contains software.

	This standard can be used in one or more of the following modes:\\

		\begin{compactitem}

			\item {\bf By an organization} - to help establish an environment of desired processes. These processes can be supported by an infrastructure of methods, procedures, techniques, tools, and trained personnel. The organization may then employ this environment to perform and manage its projects and progress systems through their life cycle stages. In this more, this book is used to assess conformance of a declared, established set of life cycle processes to its provisions. \\

			\item {\bf By a project} - to help select, structure, and employ the elements of an established set of life cycle processes to provide products and services. In this more, this book is used in the assessment of conformance of the project to the declared and established environment. \\

			\item {\bf By an acquirer and a supplier} - to help develop an agreement concerning processes and activities. Via the agreement, the processes and activities in this book are selected, negotiated, agreed to, and performed. In this mode, this book is used for guidance in developing the agreement. \\

			\item {\bf By organizations and assessors} - to perform assessments that may be used to support organizational process improvement.

		\end{compactitem}

\end{adjustwidth} 