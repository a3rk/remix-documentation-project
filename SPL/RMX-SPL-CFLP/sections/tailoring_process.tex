\section{TAILORING PROCESS}

	\subsection{PURPOSE OF THE TAILORING PROCESS}
	\begin{adjustwidth}{0.5em}{0pt}
	
		The purpose of the Tailoring Process is to adapt the processes of this standard to satisfy particular circumstances or factors that:

		\begin{compactenum}

			\item Surround an organization that is employing this standard.
			
			\item Influence a project that is required to meet an agreement.
			
			\item Reflect the needs of an organization in order to supply products or services.
		
		\end{compactenum}

	\end{adjustwidth}

	\subsection{OUTCOMES OF THE TAILORING PROCESS}
	\begin{adjustwidth}{0.5em}{0pt}

		As a result of the successful implementation of the Tailoring Process:

		\begin{compactenum}

			\item Identify and document the circumstances that influence tailoring. These influences include, but are not limited to:

			\begin{compactenum}

				\item Stability of, and variety in, operational environments.

				\item Risks, commercial or performance, to the concern of interested parties.

				\item Novelty, size and complexity.

				\item Starting date and duration of utilization.

				\item Integrity issues such as safety, security, privacy, usability, availability.

				\item Emerging technology opportunities.

				\item Profile of budget and organizational resources available.

				\item Availability of the services of enabling systems.

				\item Roles and responsibilities in the overall life cycle of the system.

				\item The need to conform to other standards.

			\end{compactenum}
			
			\item In the case of properties critical to the system, take due account of the life cycle structures recommended or mandated by standards relevant to the dimension of the criticality.

			\item Obtain input from all parties affected by the tailoring decisions. This includes, but may not be limited:

			\begin{compactenum}

				\item The system stakeholders.

				\item The interested parties to an agreement made by the organization.

				\item The contributing organizational functions.

			\end{compactenum}

			\item Make tailoring decisions in accordance with the \nameref{proc:decision_management_process} to achieve the purposes and outcomes of the selected life cycle model.

			\item Select the life cycle processes that require tailoring and delete selected outcomes, activities, or tasks.
		\end{compactenum}

	\end{adjustwidth}

	\subsection{ACTIVITIES OF THE TAILORING PROCESS}
	\begin{adjustwidth}{0.5em}{0pt}
	
		As a result of the successful implementation of the Tailoring Process:

		\begin{compactenum}

			\item Modified life cycle processes are defined to achieve the purposes and outcomes of a life cycle model.
		
		\end{compactenum}

		{\bf Notes}:

		\begin{compactitem}
			\item Organizations establish standard life cycle models as a part of the \nameref{proc:life_cycle_model_management_process}. It may be appropriate for an organization to tailor processes of this standard in order to achieve the purposes and outcomes of the stages of a life cycle model to be established.

			\item Projects select an organizationally-established life cycle model for the project as a part of the \nameref{proc:project_planning_process}. It may be appropriate to tailor organizationally-adopted processes to achieve the purposes and outcomes of the stages of the selected life cycle model.

			\item In cases where projects are directly applying this standard, it may be appropriate to tailor processes of this standard in order to achieve the purposes and outcomes of the stages of a suitable life cycle model.

			\item Irrespective of tailoring, organizations and projects are always permitted to implement processes that achieve additional outcomes or implement additional activities and tasks beyond those required for conformance to this standard.

			\item An organization or project may encounter a situation where there is the desire to modify a provision of this standard. Modification should be avoided because it may have unanticipated consequences on other processes, outcomes, activities or tasks. If necessary, modification is performed by deleting the provision (making the appropriate claim of tailored conformance) and, with careful consideration of consequences, implementing a process that achieves additional outcomes or performs additional activities and tasks beyond those of the tailored standard.

		\end{compactitem}

	\end{adjustwidth}