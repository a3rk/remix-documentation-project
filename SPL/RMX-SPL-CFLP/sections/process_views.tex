
\section{PROCESS VIEWS\label{llsec:process_views}}
\begin{adjustwidth}{0em}{0pt}

	There are instances where those representing a particular engineering interest would like to see gathered in a single place the set of process activities that directly and succinctly address their concern. For such interests, a process view can be developed to organize processes, activities, and tasks selected from this and other standards to provide a focus to their particular concern in a manner that cuts across all or parts of the life cycle. This appendix provides a process viewpoint that may be used to define process views in these instances.

\end{adjustwidth}
		

	\subsection{THE PROCESS VIEW CONCEPT\label{subsec:process_view_concept}}
	\begin{adjustwidth}{0.5em}{0pt}

		There may be cases where a unified focus is needed for activities and tasks that are selected from disparate processes to provide visibility to a significant concept or thread that cuts across the processes employed across the life cycle. It is useful to advise users of the standards how to identify and define these activities for their use, even though they cannot locate a single process that addresses their specific concern.

		For this purpose, the concept of a process view has been formulated. Like a process, the description of a process view includes a statement of purpose and outcomes. Unlike a process, the description of a process view does not include activities and tasks. Instead, the description includes guidance explaining how the outcomes can be achieved by employing the activities and tasks of the various processes in this standard. Process views can be constructed using the process viewpoint template found in the \nameref{subsubsec:process_viewpoint} sub-section below.

	\end{adjustwidth}


	\subsection{PROCESS VIEWPOINT \label{subsubsec:process_viewpoint}}
	\begin{adjustwidth}{0.5em}{0pt}

		A process view conforms to a process viewpoint. The process viewpoint provided here can be used to create process views. E.4 contains an example of applying this viewpoint.
		
		The Process Viewpoint is defined by:

		\begin{compactitem}
			
			\item its stakeholders: users of the standard;

			\item the concerns it frames: the processes needed to reflect a particular engineering interest;

			\item the contents of resulting process views should include:

			\item process view name;

			\item process view purpose;

			\item process view outcomes; and

			\item identification and description of the processes, activities and tasks which implement the process view, and references to the sources for these processes, activities and tasks in other standards.

		\end{compactitem}

	\end{adjustwidth}


	\subsection{EXAMPLE: PROCESS VIEW FOR USABILITY \label{subsubsec:example_process_view_for_usability}}
	\begin{adjustwidth}{0.5em}{0pt}

		This section provides an example of applying the process viewpoint to yield a process view for Usability, intended to illustrate how a project might assemble processes, activities and tasks of this standard to provide focused attention to the achievement of a usable product.

		This example treats the cluster of interests, generally called Usability, User centered design or Human-centered design that enables optimizing support and training, increased productivity and quality of work, improved human working conditions and reducing the chance of user rejection of the system.

		{\bf Name}: {\it Usability Process View}

	\end{adjustwidth}

		\subsubsubsection{PURPOSE}
		\begin{adjustwidth}{2em}{0pt}

			The purpose of the Usability Process View is to ensure the consideration of the interests and needs of the stakeholders in order to enable optimizing support and training, increased productivity and quality of work, improved human working conditions and reducing the chance of user rejection of the system.
			
			As a result of successful implementation of the {\it Usability Process View}:

			\begin{compactitem}

				\item the system meets the needs of users and takes account of their human capabilities and skill limitations;

				\item human factors and ergonomics knowledge and techniques are incorporated in systems design;

				\item human-centered design activities are identified and performed;

				\item system design will address possible adverse effects of use on human health, safety and performance; and

				\item systems will have enhanced user effectiveness, efficiency and satisfaction.

			\end{compactitem}

			{\bf Note}: Although involvement of users is a principle of human centered design, the outcomes permit the possibility that the desired characteristics cannot be directly measured but instead might be argued and inferred based on other product or process characteristics that can be measured.

		\end{adjustwidth}


		\subsubsubsection{PROCESSES, ACTIVITIES, TASKS}
		\begin{adjustwidth}{2em}{0pt}

			This process view can be implemented using the following processes, activities, and tasks from this standard.

			\begin{compactenum}

				\item The \nameref{proc:project_portfolio_management_process}, in particular the Process Initiation activity, provides for the establishment and maintenance of a focus on user issues in the parts of the organization that deal with markets, concept, development and support; championing of a human-centered approach.

				\item The \nameref{proc:infrastructure_management_process} provides a specification of how human-centered design activities fit into the whole systems life cycle process and the organization.

				\item The \nameref{proc:project_planning_process} provides for: selection of human centered methods and techniques, planning the involvement of users and other stakeholders, planning of human-centered design activities.

				\item The \nameref{proc:project_assessment_and_control_process} provides for monitoring the extent of achievement of the requirements and communicating the results to stakeholders and managers, ensuring a human centered approach in the design team. 

				\item The \nameref{proc:stakeholder_requirements_definition_process} provides for the identification and documentation of the context of use and the interaction between users and the system, taking into account human capabilities and skills limitations and the specification of health, safety, security, environment, training, support and other stakeholder requirements and functions that address possible adverse effects of use of the system on human health and safety.

				\item The \nameref{proc:system_requirements_analysis_process} provides for the specification and evaluation of the context of use and the usability and human centered design requirements.

				\item The \nameref{proc:system_architectural_design_process} provides for the incorporation of design criteria to address the targets for usability and the ergonomic requirements.

				\item The \nameref{proc:system_integration_process} provides for planning the integration, including the considerations for user training and the assurance that the achievement of targets for usability and accordance with ergonomic requirements are verified and recorded.

				\item The \nameref{proc:information_management_process}, in its entirety, provides for the specification, development and maintenance of artifacts for documenting and communicating the extent of achievement. 

				\item The \nameref{proc:measurement_process}, in its entirety, provides for defining an approach that relates measures to desired characteristics. 

				\item \nameref{proc:software_requirements_analysis_process} provides for the specification of the usability and software ergonomics requirements.

				\item The \nameref{proc:software_operation_process} provides for use of the system. Assuring that the usability requirements are appropriately achieved involves monitoring the operation of the system. 

				\item The \nameref{proc:software_maintenance_process} sustains the capabilities of the system, including its usability properties and can be used in its entirety.

			\end{compactenum}

		\end{adjustwidth}